\documentclass{beamer}
\mode<presentation>
\beamertemplatenavigationsymbolsempty 

\title{Debugging Numerical Software}
\author{James C. Womack}

\setbeamertemplate{footline}[text line]{%
  \strut 
  \insertauthor 
  \hfill
  \inserttitle
}

\begin{document}
\begin{frame}
  \frametitle{Bug overview}
    \begin{description}
      \item[Title] Unexpected changes to storage of local variables when compiling Fortran with OpenMP
      \item[Language] Fortran 2003/2008
      \item[Code] ONETEP (\url{http://www.onetep.org})
      \item[Submission] Attempted minimal working example (MWE)
                        with example ONETEP output files
    \end{description}
    { \footnotesize
    \begin{itemize}
      \item One-time warning is output multiple times when compiled with OpenMP
      % - The warning is in a routine which evaluates boundary conditions on the
      %   edge of a simulation cell to be passed to a multigrid solver.
      % - If non-zero charge density is too close to the box edges, a warning is 
      %   written to stdout to inform user that their charge density may be 
      %   subject to Fourier ringing near the face of the simulation cell, and 
      %   suggests acions to take.
      \item Warning should be suppressed after 1\textsuperscript{st} time by setting a static variable
      % - The warning should only be output once, during the first execution of the
      %   routine for building boundary conditions, further warning messages are
      %   unnecessary and pollute the program output.
      \item Warning \emph{is} suppressed when compiled without OpenMP
      % - As expected.
      \item Static variable uses implicit SAVE attribute via explicit initialization
      % - i.e. there is a logical variable which indicates that a warning has been
      %   issued -- this is initialized .false., but set to .true. on first output
      %   of the message.
      % - The variable is static between calls to the boundary condition routine
      %   because it has an implicit SAVE attribute
      %   --> It is made static because it has been explicitly initialized during
      %       type declaration (a common practice in Fortran).
      % - Implicit SAVE attribute via explicit initialization is part of the 
      %   Fortran standard (as far as I know...), so we expect it to be respected.
      % - If we compile with ifort with -qopenmp flag, then this variable 
      %   appears to consistently loose its value.
      \item Occurs with Intel Fortran but not GFortran
      % - Regardless of whether compiled with -fopenmp flag or not, the static
      %   variable maintains its value between calls to the boundary condition
      %   evaluation routine.
      \item Adding explicit SAVE attribute solves the issue with OpenMP, \emph{but why?}
      % - This is unexpected -- is ifort not respecting our implicit SAVE 
      %   attribute?
      \item A MWE has so far proven elusive...
      % - Things turn out to be more complicated than "turning on OpenMP support
      %   in ifort causes implicitly SAVEd variables to not be statically 
      %   allocated".
      % - The issue seems to depend on other factors.
    \end{itemize}
    }
\end{frame}
\end{document}
